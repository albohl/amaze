\documentclass[12pt, a4paper]{article}
\usepackage[english]{babel}
\usepackage[utf8]{inputenc}
\usepackage{hyperref}
\usepackage{listings}
\usepackage{mathtools}
\usepackage[ top=1.7cm, bottom=1.7cm, left=2cm, right=2cm ]{geometry}
\begin{document}
\title{Amazing Maze - Documentation of Group Project, PKD 13/14}
\author{Victoria Catalán, Albin Ohlsson \& Anton Strindell}
\maketitle
\tableofcontents
\lstset{breaklines=true }
\lstset{
	language=ML,
	basicstyle=\footnotesize\ttfamily,
}

\section{instructions}
You must write documentation for your program. Apart from the standard specifications for each function to be included in the application code, you must write a separate document that describes how your program works and how to use it. 
The documentation needs to include :

    The title, names of the project participants, a statement of the fact that this is the project for this course this year, a table of contents, an introduction, and summary of what the program does.
    Use cases: a guide for how to actually use your program, including key examples.
    Program documentation: a description of how your program really works, including at least: 
        Description of data structures. For abstract data types, you should also describe the interface.
        Description of the algorithms your program uses.
        Description of the various functions of the program. Describe algorithms and provide functional specifications for the main elements. Talk about how the program flow looks (i.e., how functions call each other).
    Description of known shortcomings of the program. There may be things that work but not as well as you would like, or things that despite valiant attempts you have not succeeded in implementing properly.

It should be sufficient to read the documentation to understand your program!

Documentation is as important as the program itself. Poor documentation will affect your grade even if your program is working well. Start working on the documentation in good time! For example, work out the general structure of your document, e.g., section headings or table of contents, that will show how you intend to organise the documentation, and show this to your supervisor.

\section{About}
Amazing Maze is a program that both generates and solves labyrinths. The mazes created can be of varying sizes and difficulties. The user can then try to solve them or use the program to get the solution.

\section{Use cases}


\end{document}